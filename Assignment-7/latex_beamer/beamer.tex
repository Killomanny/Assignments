\documentclass{beamer}
\usepackage{listings}


\lstset{
%language=C,
frame=single, 
breaklines=true,
columns=fullflexible
}


\usepackage{subcaption}
\usepackage{url}
\usepackage{tikz}
\usepackage{tkz-euclide} 
\usetikzlibrary{calc,math}
\usepackage{float}
\usepackage[export]{adjustbox}
\usepackage[utf8]{inputenc}
\usepackage{amsmath}
\usepackage[version=4]{mhchem}


\newcounter{saveenumi}


\newcommand\norm[1]{\left\lVert#1\right\rVert}
\newcommand{\myvec}[1]{\ensuremath{\begin{pmatrix}#1\end{pmatrix}}}
\newcommand{\seti}{\setcounter{saveenumi}{\value{enumi}}}
\newcommand{\conti}{\setcounter{enumi}{\value{saveenumi}}}


\providecommand{\brak}[1]{\ensuremath{\left(#1\right)}}
\providecommand{\cbrak}[1]{\ensuremath{\left\{#1\right\}}}
\providecommand{\lcbrak}[1]{\ensuremath{\left\{#1\right.}}
\providecommand{\rcbrak}[1]{\ensuremath{\left.#1\right\}}}
\providecommand{\pr}[1]{\ensuremath{\Pr\left(#1\right)}}
\providecommand{\sbrak}[1]{\ensuremath{{}\left[#1\right]}}
\providecommand{\lsbrak}[1]{\ensuremath{{}\left[#1\right.}}
\providecommand{\rsbrak}[1]{\ensuremath{{}\left.#1\right]}}


\renewcommand{\vec}[1]{\mathbf{#1}}


\usetheme{Boadilla}


\title{ Assignment-6 ( Cbse 12 ex 13.2 12 )}
\author{Asli}
\institute{IIT Hyderabad}
\date{\today }



\begin{document}



\begin{frame}
      \titlepage
\end{frame}



\begin{frame}{Abstract}
     \begin{block}{Abstract}
          This document contains the solution to Cbse class $12$ ex $13.4 8$
      \end{block}
\end{frame}



\begin{frame}{Problem}
       \begin{block}{Question}
           A random variable $X$ has the following probability distribution:
           
               \begin{table}[!htb]
                     \centering
                            {
                                   \begin{tabular}{|c|c|c|c|c|c|c|c|c|}
                                       \hline
                                         $X$ & $0$ & $1$ & $2$ & $3$ & $4$ & $5$ & $6$ & $7$  \\ \hline
                                      $P(X)$ & $0$ & $k$ & $2k$ & $2k$ & $3k$ & $k^{2}$ & $2 k^{2}$ &    $7k^{2}+k$ \\ \hline
                                   \end{tabular}
                              }
                         \caption{ Probability Distribution }
                         \label{tab:table}
              \end{table}
              
              determine
              
              \begin{enumerate}
                  \item $k$
                  \item $P(X < 3)$
                  \item $P(X > 6)$
                  \item $P(0 < X < 3)$
              \end{enumerate}
                          
       \end{block}
       
\end{frame}



\begin{frame}{Theory}

        \begin{block}{Property required 1}
                The sum of all the values of $P(X)$ should always sum upto one.
                
                       \begin{align}
                              \sum^{n}_{i = 1}p_i = 1, \text{for $i = {1,2,3,...,n}$}
                              \label{eq:form1}
                       \end{align}
        \end{block}
        
        \begin{block}{Property required 1}
                 The value of P(X) should always be positive.
                 
                       \begin{align}
                              p_i>0, \text{for $i = {1,2,3,...,n}$}
                              \label{eq:form2}
                       \end{align}
                       
        \end{block}
        
\end{frame}



\begin{frame}{Solution 1}

       \begin{block}{determining $k$}
                with help of Eq\eqref{eq:form1} , Eq\eqref{eq:form2} and based on values on Table\ref{tab:table} we can write
                
                        \begin{align}
                           &\implies 0 + k + 2k + 2k + 3k + k^{2} + 2k^{2} + ( 7k^{2} + k )   =  1 \\
                           &\implies 10k^{2} + 9k = 1 \\
                           &\implies 10k^{2} + 9k - 1 \\
                           &\implies \brak{10k - 1}\brak{k+1} 
                        \end{align}
       \end{block}

\end{frame}



\begin{frame}{Solution 1}

       \begin{block}{continuation of determining $k$}
                               So $k$ can have values $\frac{1}{10}$ , $-1$ but we know based on Eq\eqref{eq:form2} that probability cannot be negative \\
                               $\therefore$ the value of $k$ is $0.1$
       \end{block}

\end{frame}



\begin{frame}{Solution 2}

       \begin{block}{finding value of $P(X < 3)$}
       
                \begin{align}
                    P(X < 3 ) &= P(X = 0 ) + P(X = 1 ) + P(X = 2 ) \\
                              &= 0 + k + 2k \\
                              &= 3k \\
                              &= 0.3 
                \end{align}
                
                $\therefore$ the value 0f $P(X < 3 )$ is $0.30$
       \end{block}

\end{frame}



\begin{frame}{Solution 3}

       \begin{block}{finding value of $P(X > 6)$}
       
                \begin{align}
                    P(X > 6 ) &= P(X = 7 )  \\
                              &= 7k^{2} + k \\
                              &= 7\brak{0.1}^{2} + 0.1 \\
                              &= 0.07 + 0.1 \\
                              &= 0.17 
                \end{align}
                
                $\therefore$ the value 0f $P(X > 6 )$ is $0.17$
       \end{block}

\end{frame}



\begin{frame}{Solution 4}
       \begin{block}{finding value of $P(0 < X < 3)$}
       
                \begin{align}
                    P(0 < X < 3 )  &= P(X = 1 ) + P(X = 2 ) \\
                                   &= k + 2k \\
                                   &= 3k \\
                                   &= 0.3
                \end{align}
                
                $\therefore$ the value 0f $P(0 < X < 3)$ is $0.30$
       \end{block}

\end{frame}


         
\end{document}
