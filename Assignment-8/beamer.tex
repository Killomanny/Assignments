\documentclass{beamer}
\usepackage{listings}


\lstset{
%language=C,
frame=single, 
breaklines=true,
columns=fullflexible
}


\usepackage{subcaption}
\usepackage{url}
\usepackage{tikz}
\usepackage{tkz-euclide} 
\usetikzlibrary{calc,math}
\usepackage{float}
\usepackage[export]{adjustbox}
\usepackage[utf8]{inputenc}
\usepackage{amsmath}
\usepackage[version=4]{mhchem}


\newcounter{saveenumi}


\newcommand\norm[1]{\left\lVert#1\right\rVert}
\newcommand{\myvec}[1]{\ensuremath{\begin{pmatrix}#1\end{pmatrix}}}
\newcommand{\seti}{\setcounter{saveenumi}{\value{enumi}}}
\newcommand{\conti}{\setcounter{enumi}{\value{saveenumi}}}


\providecommand{\brak}[1]{\ensuremath{\left(#1\right)}}
\providecommand{\cbrak}[1]{\ensuremath{\left\{#1\right\}}}
\providecommand{\lcbrak}[1]{\ensuremath{\left\{#1\right.}}
\providecommand{\rcbrak}[1]{\ensuremath{\left.#1\right\}}}
\providecommand{\pr}[1]{\ensuremath{\Pr\left(#1\right)}}
\providecommand{\sbrak}[1]{\ensuremath{{}\left[#1\right]}}
\providecommand{\lsbrak}[1]{\ensuremath{{}\left[#1\right.}}
\providecommand{\rsbrak}[1]{\ensuremath{{}\left.#1\right]}}


\renewcommand{\vec}[1]{\mathbf{#1}}


\usetheme{Boadilla}


\title{ Assignment-8 Papoullis chapter 4 ex 17 que}
\author{Asli}
\institute{IIT Hyderabad}
\date{\today }



\begin{document}



\begin{frame}
      \titlepage
\end{frame}



\begin{frame}{Abstract}
     \begin{block}{Abstract}
          This document contains the solution to Papoullis chapter 4 exercise 17 question
      \end{block}
\end{frame}



\begin{frame}{Problem}

          \begin{block}{Problem}
              Show that if $\beta(t)$ = $f(t| x > t)$ is the conditional failure rate of the random variable x and $\beta(t) = kt $, then $f(x)$ is a Rayleigh density
           \end{block}

\end{frame}



\begin{frame}{Solution}

            \begin{block}{Given things}
                 
                    given $\beta(t)$ = $f(t| x > t)$ is the conditional failure rate of the random variable x and $\beta(t) = kt $
                  
                    \begin{align}
                         \beta(t) &= f(t| x > t) \\
                                  &= \frac{f(t)}{1 - F(t)} \\
                                  &= \frac{ \frac{d}{dt}F(t) }{1 - F(t)}
                                  \label{eq:form1}
                    \end{align}
                
            \end{block}
    
\end{frame}



\begin{frame}{Solution}

            \begin{block}{Continuation}
            
                  on substituting $\beta(t)$ in Eq\eqref{eq:form1} we will get below thing
                  
                    \begin{align}
                         kt &= \frac{ \frac{d}{dt}F(t) }{1 - F(t)} \\
                            &= \frac{d}{dt}\brak{ -\log[1-F(t)]} 
                    \end{align}
                    
                  Integrating both sides of this equation from 0 to x yields


                
            \end{block}
    
\end{frame}



\begin{frame}{Solution}
   
            \begin{block}{Continuation}
               
                as $\beta(t) = kt > 0 $ so t cannot be negative so we can say like for all $F(x)$ where $x < 0 $ or $ x = 0 $ the value is $0$ \\
                also for integration we take limits from $0$ to $t$ for the same reason 
            
            \end{block}
    
\end{frame}



\begin{frame}{Solution}

            \begin{block}{Continuation}
                  
                    \begin{align}
                       \int_{0}^{x}(kt) dt = \brak{-\log[1-F(x)]} - \brak{-\log[1-F(0)]} 
                       \label{eq:form2}
                    \end{align}
                    
                    since $ F(0)  = 0 $ we can write and simplyfy Eq\eqref{eq:form2} as
                    
                    \begin{align}
                        \frac{k(x)^{2}}{2} = -\log[1-F(x)] 
                        \label{eq:form3}
                    \end{align}
                
            \end{block}
    
\end{frame}



\begin{frame}{Solution}

            \begin{block}{Continuation}
                  
                  on simplyfying Eq\eqref{eq:form3} we will get
                  
                    \begin{align}
                        1-F(x) = {e}^{-\frac{k(x)^{2}}{2}} 
                        \label{eq:form4}
                    \end{align}
                    
                  on further simplyfying Eq\eqref{eq:form4} we will get
                  
                    \begin{align}
                        F(x) = 1 - {e}^{-\frac{k(x)^{2}}{2}} 
                        \label{eq:form5}
                    \end{align}
                
            \end{block}
    
\end{frame}



\begin{frame}{Solution}

            \begin{block}
                  
                  on Diffrentiating Eq\eqref{eq:form5} we will get
                  
                    \begin{align}
                         f(x) =  kx\brak{e^{-\frac{k(x)^{2}}{2}}} \text{ for $ x > 0 $}
                    \end{align}
                    
                    which is nothing but in the form of Rayleigh density
                    
            \end{block}
    
\end{frame}

         
\end{document}
