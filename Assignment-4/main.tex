\let\negmedspace\undefined
\let\negthickspace\undefined
\documentclass[journal,12pt,twocolumn]{IEEEtran}

\usepackage{csquotes}
\usepackage{comment}
\usepackage{enumerate}
\usepackage{amsmath,amssymb,amsthm}
\usepackage{graphicx}
\let\vec\mathbf
\newcommand{\myvec}[1]{\ensuremath{\begin{pmatrix}#1\end{pmatrix}}}
\providecommand{\brak}[1]{\ensuremath{\left(#1\right)}}
\providecommand{\pr}[1]{\ensuremath{\Pr\left(#1\right)}}



\title{Assignment 4 (CBSE Class 11 Ex 16.4 Example 12	 }
\author{Busireddy Asli Nitej Reddy (CS21BTECH11011)}
\date{}



\begin{document}



\maketitle



\begin{abstract}
This document contains the solution to Example 12 of Chapter 16 (Probability) in the CBSE Class 11.
\end{abstract}



\section*{\textbf{Problem}}

  Two students Anil and Ashima appeared in an examination. The probability that Anil will qualify the examination is $0.05$ and that Ashima will qualify the examination is $0.10$. The probability that both will qualify the examination is $0.02$. Find the probability that
     \begin{enumerate}[a)]
         \item Both Anil and Ashima will not qualify the examination.
         \item Atleast one of them will not qualify the examination and
         \item Only one of them will qualify the examination.
     \end{enumerate}



\section*{\textbf{Solution}}

let A and B denote two events where A denotes Anil passes and B denotes Ashima passes the examination given $\pr{A} = 0.05, \pr{B} = 0.10 $ and $\pr{A B} = 0.02 $. 



\begin{enumerate}




     \item[(i)] Let the event ‘both Anil and Ashima will not qualify the examination’ be T and probability for that is denoted as $\pr{T}$ and it is expressed as $T = A^{\prime}B^{\prime} =  \brak{A+B}^{\prime}$

          \begin{align}
              \pr{T} &=  \pr{\brak{A+B}^{\prime}}  \\
              \label{eq:form1}
                     &= 1 - \pr{A+B}
          \end{align} 
    
     and for calculating value of $\pr{A+B}$ we use
    
          \begin{multline}
                \label{eq:sum}
                      \pr{A+B} = \pr{A} + \pr{A} \\
                      - \pr{A B}
          \end{multline}
    
    on substituting value's in eq\eqref{eq:sum} will get
    
          \begin{align}
                \pr{A+B}  &=  0.05 + 0.10 - 0.02 \\
                          &=  0.13
          \end{align}
    
    resubstituting this in eq\eqref{eq:form1} we will get
    
          \begin{align}
                \pr{T} &= 1 - 0.13 \\
                       &= 0.87  
          \end{align} 
    
    $\therefore$ the probability that both Anil and Ashima will not qualify the examination is $0.87$
     
     
     
     
     \item[(ii)] Let the event where Atleast one of them will not qualify the examination be K and the probability is given by $\pr{K}$
     
          \begin{align}
                \pr{K} &= \pr{A^{\prime}+B^{\prime}} \\
                       &= \pr{\brak{A B}^{\prime}} \\
                       &= 1 - \pr{A B}
          \end{align}
    
    on substituting value 
    
          \begin{align}
               \pr{K} &= 1 - 0.02  \\
                      &= 0.98 
          \end{align}
    
    $\therefore$ the probability that atleast one of them will not qualify the examination is $0.98$
    
    
    
    
    \item[(iii)] Let the event where Only one of them will qualify the examination be D and probability of that event is given by $\pr{D}$  and is given by
    
           \begin{align}
               \label{eq:form3}
               \pr{D} = \pr{\brak{A B^{\prime}} + \brak{A^{\prime} B}}   
           \end{align}
    
    as both events are mutually exclusive because both cannot exist together one with Anil pass other with Anil fail same for Ashima 
    
            \begin{align}
                \brak{A B^{\prime}}\brak{A^{\prime} B} = 0, \because BB^{\prime} = 0
                \label{eq:axiom_sum_AB}
            \end{align}
            
    Hence, $ \brak{A B^{\prime}} $ and $ \brak{A^{\prime} B}  $ are mutually exclusive \\  
    so we can simplyfy the eq\eqref{eq:form3} further more as  
    
            \begin{align}
                     \label{eq:form6}
                     \pr{D} = \pr{\brak{A B^{\prime}}} + \pr{\brak{A^{\prime} B}}
            \end{align}
    
    further $ \pr{A B^{\prime}} $ and $ \pr{A^{\prime} B} $ can be simplified as
    
            \begin{align}
                  A = A \brak{B+B^{\prime}} =  AB + AB^{\prime}
                 \label{eq:axiom_sum_A}
            \end{align}
    
    and 
    
            \begin{align}
                  \brak{ AB}\brak{  AB^{\prime}} = 0, \because BB^{\prime} = 0
                  \label{eq:axiom_sum_ABP}
                  \end{align}
                  
    Hence, $AB$ and $AB^{\prime}$ are mutually exclusive and 
    
            \begin{align}
                     \pr{A} = \pr{AB} + \pr{AB^{\prime}} \\
                  \implies 
                     \pr{A B^{\prime}} =  \pr{A} - \pr{AB}
                     \label{eq:axiom_sum_ABPP}
            \end{align}
    
    similarly we can also get for 
    
            \begin{align}
                   \pr{A^{\prime} B} =  \pr{B} - \pr{AB}
                   \label{eq:axiom_sum_ABPPP}
            \end{align}
    
    substituting eq\eqref{eq:axiom_sum_ABPPP} and eq\eqref{eq:axiom_sum_ABPP} in eq\eqref{eq:form6} we will get
    
            \begin{multline}
                  \label{eq:sum2}
                   \pr{D} = \pr{A} - \pr{AB} 
                 \\
                   +\pr{B} - \pr{AB}
            \end{multline}
    
    on substituting value we will get 
    
            \begin{align}
                 \pr{D} &=  0.05 - 0.02 + 0.10 - 0.02 \\
                        &= 0.11
            \end{align}    
    
    $\therefore$ the probability where Only one of them will qualify the examination is $0.11$
    


\end{enumerate}



\end{document}