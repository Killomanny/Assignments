\documentclass[journal,12pt,twocolumn]{IEEEtran}
\usepackage{setspace}
\usepackage{gensymb}
\usepackage{caption}
%\usepackage{multirow}
%\usepackage{multicolumn}
%\usepackage{subcaption}
%\doublespacing
\singlespacing
\usepackage{csvsimple}
\usepackage{amsmath}
\usepackage{multicol}
%\usepackage{enumerate}
\usepackage{amssymb}
%\usepackage{graphicx}
\usepackage{newfloat}
%\usepackage{syntax}
\usepackage{listings}
\usepackage{color}
\usepackage{tikz}
\usetikzlibrary{shapes,arrows}



%\usepackage{graphicx}
%\usepackage{amssymb}
%\usepackage{relsize}
%\usepackage[cmex10]{amsmath}
%\usepackage{mathtools}
%\usepackage{amsthm}
%\interdisplaylinepenalty=2500
%\savesymbol{iint}
%\usepackage{txfonts}
%\restoresymbol{TXF}{iint}
%\usepackage{wasysym}
\usepackage{amsthm}
\usepackage{mathrsfs}
\usepackage{txfonts}
\usepackage{stfloats}
\usepackage{cite}
\usepackage{cases}
\usepackage{mathtools}
\usepackage{caption}
\usepackage{enumerate}	
\usepackage{enumitem}
\usepackage{amsmath}
%\usepackage{xtab}
\usepackage{longtable}
\usepackage{multirow}
%\usepackage{algorithm}
%\usepackage{algpseudocode}
\usepackage{enumitem}
\usepackage{mathtools}
\usepackage{hyperref}
%\usepackage[framemethod=tikz]{mdframed}
\usepackage{listings}
    %\usepackage[latin1]{inputenc}                                 %%
    \usepackage{color}                                            %%
    \usepackage{array}                                            %%
    \usepackage{longtable}                                        %%
    \usepackage{calc}                                             %%
    \usepackage{multirow}                                         %%
    \usepackage{hhline}                                           %%
    \usepackage{ifthen}                                           %%
  %optionally (for landscape tables embedded in another document): %%
    \usepackage{lscape}     


\usepackage{url}
\def\UrlBreaks{\do\/\do-}


%\usepackage{stmaryrd}


%\usepackage{wasysym}
%\newcounter{MYtempeqncnt}
\DeclareMathOperator*{\Res}{Res}
%\renewcommand{\baselinestretch}{2}
\renewcommand\thesection{\arabic{section}}
\renewcommand\thesubsection{\thesection.\arabic{subsection}}
\renewcommand\thesubsubsection{\thesubsection.\arabic{subsubsection}}

\renewcommand\thesectiondis{\arabic{section}}
\renewcommand\thesubsectiondis{\thesectiondis.\arabic{subsection}}
\renewcommand\thesubsubsectiondis{\thesubsectiondis.\arabic{subsubsection}}

% correct bad hyphenation here
\hyphenation{op-tical net-works semi-conduc-tor}

%\lstset{
%language=C,
%frame=single, 
%breaklines=true
%}

%\lstset{
	%%basicstyle=\small\ttfamily\bfseries,
	%%numberstyle=\small\ttfamily,
	%language=Octave,
	%backgroundcolor=\color{white},
	%%frame=single,
	%%keywordstyle=\bfseries,
	%%breaklines=true,
	%%showstringspaces=false,
	%%xleftmargin=-10mm,
	%%aboveskip=-1mm,
	%%belowskip=0mm
%}

%\surroundwithmdframed[width=\columnwidth]{lstlisting}
\def\inputGnumericTable{}                                 %%
\lstset{
%language=C,
frame=single, 
breaklines=true,
columns=fullflexible
}
 

\begin{document}
%
\tikzstyle{block} = [rectangle, draw,
    text width=3em, text centered, minimum height=3em]
\tikzstyle{sum} = [draw, circle, node distance=3cm]
\tikzstyle{input} = [coordinate]
\tikzstyle{output} = [coordinate]
\tikzstyle{pinstyle} = [pin edge={to-,thin,black}]
\providecommand{\e}[1]{\ensuremath{E\left(#1\right)}}
\providecommand{\es}[1]{\ensuremath{E\left[#1\right]}}
\theoremstyle{definition}
\newtheorem{theorem}{Theorem}[section]
\newtheorem{problem}{Problem}
\newtheorem{proposition}{Proposition}[section]
\newtheorem{lemma}{Lemma}[section]
\newtheorem{corollary}[theorem]{Corollary}
\newtheorem{example}{Example}[section]
\newtheorem{definition}{Definition}[section]
%\newtheorem{algorithm}{Algorithm}[section]
%\newtheorem{cor}{Corollary}
\newcommand{\BEQA}{\begin{eqnarray}}
\newcommand{\EEQA}{\end{eqnarray}}
\newcommand{\define}{\stackrel{\triangle}{=}}
\bibliographystyle{IEEEtran}
%\bibliographystyle{ieeetr}
\providecommand{\nCr}[2]{\,^{#1}C_{#2}} % nCr
\providecommand{\nPr}[2]{\,^{#1}P_{#2}} % nPr
\providecommand{\mbf}{\mathbf}
\providecommand{\pr}[1]{\ensuremath{\Pr\left(#1\right)}}
\providecommand{\qfunc}[1]{\ensuremath{Q\left(#1\right)}}
\providecommand{\sbrak}[1]{\ensuremath{{}\left[#1\right]}}
\providecommand{\lsbrak}[1]{\ensuremath{{}\left[#1\right.}}
\providecommand{\rsbrak}[1]{\ensuremath{{}\left.#1\right]}}
\providecommand{\brak}[1]{\ensuremath{\left(#1\right)}}
\providecommand{\lbrak}[1]{\ensuremath{\left(#1\right.}}
\providecommand{\rbrak}[1]{\ensuremath{\left.#1\right)}}
\providecommand{\cbrak}[1]{\ensuremath{\left\{#1\right\}}}
\providecommand{\lcbrak}[1]{\ensuremath{\left\{#1\right.}}
\providecommand{\rcbrak}[1]{\ensuremath{\left.#1\right\}}}
\theoremstyle{remark}
\newtheorem{rem}{Remark}
\newcommand{\sgn}{\mathop{\mathrm{sgn}}}
% \providecommand{\abs}[1]{\left\vert#1\right\vert}
% \providecommand{\res}[1]{\Res\displaylimits_{#1}} 
% \providecommand{\norm}[1]{\left\Vert#1\right\Vert}
% \providecommand{\mtx}[1]{\mathbf{#1}}
% \providecommand{\mean}[1]{E\left[ #1 \right]}
\providecommand{\fourier}{\overset{\mathcal{F}}{ \rightleftharpoons}}
%\providecommand{\hilbert}{\overset{\mathcal{H}}{ \rightleftharpoons}}
\providecommand{\system}{\overset{\mathcal{H}}{ \longleftrightarrow}}
	%\newcommand{\solution}[2]{\textbf{Solution:}{#1}}
\newcommand{\solution}{\noindent \textbf{Solution: }}
\newcommand{\myvec}[1]{\ensuremath{\begin{pmatrix}#1\end{pmatrix}}}
\providecommand{\dec}[2]{\ensuremath{\overset{#1}{\underset{#2}{\gtrless}}}}
\DeclarePairedDelimiter{\ceil}{\lceil}{\rceil}
%\numberwithin{equation}{section}
%\numberwithin{problem}{subsection}
%\numberwithin{definition}{subsection}
\makeatletter
\@addtoreset{figure}{section}
\makeatother
\let\StandardTheFigure\thefigure
%\renewcommand{\thefigure}{\theproblem.\arabic{figure}}
\renewcommand{\thefigure}{\thesection}
%\numberwithin{figure}{subsection}
%\numberwithin{equation}{subsection}
%\numberwithin{equation}{section}
%\numberwithin{equation}{problem}
%\numberwithin{problem}{subsection}
\numberwithin{problem}{section}
%%\numberwithin{definition}{subsection}
%\makeatletter
%\@addtoreset{figure}{problem}
%\makeatother
\makeatletter
\@addtoreset{table}{section}
\makeatother
\let\StandardTheFigure\thefigure
\let\StandardTheTable\thetable
\let\vec\mathbf
\numberwithin{equation}{section}
\vspace{3cm}
\title{%Convex Optimization in Python
	{
	RANDOM NUMBERS
	}
}
%\title{
%	\logo{Matrix Analysis through Octave}{\begin{center}\includegraphics[scale=.24]{tlc}\end{center}}{}{HAMDSP}
%}
% paper title
% can use linebreaks \\ within to get better formatting as desired
%\title{Matrix Analysis through Octave}
%
%
% author names and IEEE memberships
% note positions of commas and nonbreaking spaces ( ~ ) LaTeX will not break
% a structure at a ~ so this keeps an author's name from being broken across
% two lines.
% use \thanks{} to gain access to the first footnote area
% a separate \thanks must be used for each paragraph as LaTeX2e's \thanks
% was not built to handle multiple paragraphs
%
\author{ BUSIREDDY ASLI NITEJ REDDY }
% note the % following the last \IEEEmembership and also \thanks - 
% these prevent an unwanted space from occurring between the last author name
% and the end of the author line. i.e., if you had this:
% 
% \author{....lastname \thanks{...} \thanks{...} }
%                     ^------------^------------^----Do not want these spaces!
%
% a space would be appended to the last name and could cause every name on that
% line to be shifted left slightly. This is one of those "LaTeX things". For
% instance, "\textbf{A} \textbf{B}" will typeset as "A B" not "AB". To get
% "AB" then you have to do: "\textbf{A}\textbf{B}"
% \thanks is no different in this regard, so shield the last } of each \thanks
% that ends a line with a % and do not let a space in before the next \thanks.
% Spaces after \IEEEmembership other than the last one are OK (and needed) as
% you are supposed to have spaces between the names. For what it is worth,
% this is a minor point as most people would not even notice if the said evil
% space somehow managed to creep in.
% The paper headers
%\markboth{Journal of \LaTeX\ Class Files,~Vol.~6, No.~1, January~2007}%
%{Shell \MakeLowercase{\textit{et al.}}: Bare Demo of IEEEtran.cls for Journals}
% The only time the second header will appear is for the odd numbered pages
% after the title page when using the twoside option.
% 
% *** Note that you probably will NOT want to include the author's ***
% *** name in the headers of peer review papers.                   ***
% You can use \ifCLASSOPTIONpeerreview for conditional compilation here if
% you desire.
% If you want to put a publisher's ID mark on the page you can do it like
% this:
%\IEEEpubid{0000--0000/00\$00.00~\copyright~2007 IEEE}
% Remember, if you use this you must call \IEEEpubidadjcol in the second
% column for its text to clear the IEEEpubid mark.
% make the title area
\maketitle
\tableofcontents
\bigskip
\renewcommand{\thefigure}{\theenumi}
\renewcommand{\thetable}{\theenumi}
\begin{abstract}
This manual provides solutions to the Assignment on Random Numbers
\end{abstract}
%template ends here
\section{Uniform Random Numbers}
Let $U$ be a uniform random variable between 0 and 1.
\begin{enumerate}[label=\thesection.\arabic*
,ref=\thesection.\theenumi]
\item Generate $10^6$ samples of $U$ using a C program and save into a file called uni.dat .
\\
\solution Download the following files and execute the  C program.
\begin{lstlisting}
wget https://github.com/Killomanny/Assignments/blob/main/Assignmets2.0/codes/1.1.c
wget https://github.com/Killomanny/Assignments/blob/main/Assignmets2.0/codes/source.h
\end{lstlisting}
Download the above files and execute the following commands
\begin{lstlisting}
$ gcc 1.1.c
$ ./a.out
\end{lstlisting}
\item
Load the uni.dat file into python and plot the empirical CDF of $U$ using the samples in uni.dat. The CDF is defined as
\begin{align}
F_{U}(x) = \pr{U \le x}
\end{align}
\\
\solution  The following code plots Fig. \ref{fig:1.2}
\begin{lstlisting}
wget https://github.com/Killomanny/Assignments/blob/main/Assignmets2.0/codes/1.2.py
\end{lstlisting}
Download the above files and execute the following commands to produce Fig.\ref{fig:1.2}
\begin{lstlisting}
$ python3 1.2.py
\end{lstlisting}
\begin{figure}[!h]
\centering
\includegraphics[width=\columnwidth]{./figs/1.2.png}
\caption{The CDF of $U$}
\label{fig:1.2}
\end{figure}
%
\item
Find a  theoretical expression for $F_{U}(x)$.\\
\solution Given $U$ is a uniform Random Variable
\begin{align}
p_{U}(x)=1 \text{ for } 0 < x < 1 \\
F_U(x)=\int_{-\infty}^{\infty}p_{U}(x)dx\\
\boxed{\implies F_U(x)=
\begin{cases}
 0 &x\le0\\
 x &0< x< 1\\
 1 &x\ge 1
\end{cases}}
\end{align}
\begin{lstlisting}
wget https://github.com/Killomanny/Assignments/blob/main/Assignmets2.0/codes/1.3.py
\end{lstlisting}
\item
The mean of $U$ is defined as
%
\begin{equation}
E\sbrak{U} = \frac{1}{N}\sum_{i=1}^{N}U_i
\end{equation}
%
and its variance as
%
\begin{equation}
\text{var}\sbrak{U} = E\sbrak{U- E\sbrak{U}}^2 
\end{equation}
Write a C program to  find the mean and variance of $U$. \\
\solution Download the following files and execute the  C program.
\begin{lstlisting}
wget https://github.com/Killomanny/Assignments/blob/main/Assignmets2.0/codes/1.4.c
wget https://github.com/Killomanny/Assignments/blob/main/Assignmets2.0/codes/source.h
\end{lstlisting}
Download the above files and execute the following commands
\begin{lstlisting}
$ gcc 1.4.c
$ ./a.out
\end{lstlisting}
\item Verify your result theoretically given that
\end{enumerate}
%
\begin{equation}
E\sbrak{U^k} = \int_{-\infty}^{\infty}x^kdF_{U}(x)
\end{equation}
\solution 
\begin{align}
    \text{var}\sbrak{U} &= E\sbrak{U- E\sbrak{U}}^2\\ 
    \implies \text{var}\sbrak{U} &= E\sbrak{U^2}- E\sbrak{U}^2 \\
    E\sbrak{U}&=\int_{-\infty}^{\infty}xdF_U(x)\\
    E\sbrak{U}&=\int_{0}^{1}x\\
    \implies \boxed{E\sbrak{U}=\frac{1}{2}}\\
    E\sbrak{U^2}&=\int_{-\infty}^{\infty}x^{2}dF_U(x)\\
    E\sbrak{U^2}&=\int_{0}^{1}x^{2}dF_U(x)\\
    \implies E\sbrak{U^2}&=\frac{1}{3}\\
    \implies \boxed{\text{var}\sbrak{U}=\frac{1}{12}}
\end{align}
Theoretical values 
\begin{align}
E(X) &= 0.5 \\
Var(X) &= 0.08333
\end{align}
Numerical values calculated in C program
\begin{align}
E(X) = 0.500007\\
Var(X) = 0.083307
\end{align}

\section{Central Limit Theorem}
%
\begin{enumerate}[label=\thesection.\arabic*
,ref=\thesection.\theenumi]
%
\item
Generate $10^6$ samples of the random variable
%
\begin{equation}
X = \sum_{i=1}^{12}U_i -6
\end{equation}
%
using a C program, where $U_i, i = 1,2,\dots, 12$ are  a set of independent uniform random variables between 0 and 1 and save in a file called gau.dat\\
\solution Download the following files and execute the  C program.
\begin{lstlisting}
wget https://github.com/Killomanny/Assignments/blob/main/Assignmets2.0/codes/2.1.c
wget https://github.com/Killomanny/Assignments/blob/main/Assignmets2.0/codes/source.h
\end{lstlisting}
Download the above files and execute the following commands
\begin{lstlisting}
$ gcc 2.1.c
$ ./a.out
\end{lstlisting}
\item
Load gau.dat in python and plot the empirical CDF of $X$ using the samples in gau.dat. What properties does a CDF have?\\
\solution The CDF of $X$ is plotted using the code below
\begin{lstlisting}
wget https://github.com/Killomanny/Assignments/blob/main/Assignmets2.0/codes/2.2.py
\end{lstlisting}
run the following code to get the graph
\begin{lstlisting}
$ python3 2.2.py
\end{lstlisting}
Some of the properties of CDF 
\begin{enumerate}
\item $\lim_{x \to \infty}F_X(x) = 1$
    \item $F_X(x)$ is non decreasing function.
    \item Symmetric about one point.
\end{enumerate}
\item
Load gau.dat in python and plot the empirical PDF of $X$ using the samples in gau.dat. The PDF of $X$ is defined as
\begin{align}
p_{X}(x) = \frac{d}{dx}F_{X}(x)
\end{align}
What properties does the PDF have?
\\
\solution The PDF of $X$ is plotted using the code below
\begin{lstlisting}
wget https://github.com/Killomanny/Assignments/blob/main/Assignmets2.0/codes/2.3.py
\end{lstlisting}
Download the above files and execute the following commands to produce the PDF
\begin{lstlisting}
$ python3 2.3.py
\end{lstlisting}
Some of the properties of the PDF:
\begin{enumerate}
    \item Symmetric about $x=\mu$ in this case
    \item Decreasing function for $x>\mu$ and increasing for $x<\mu$ and attains maximum at $x=\mu$
    \item Area under the curve is unity.
    \item the PDF takes non negative values
\end{enumerate}
\item Find the mean and variance of $X$ by writing a C program.\\
\solution Download the following files and execute the  C program.
\begin{lstlisting}
wget https://github.com/Killomanny/Assignments/blob/main/Assignmets2.0/codes/2.4.c
wget https://github.com/Killomanny/Assignments/blob/main/Assignmets2.0/codes/source.h
\end{lstlisting}
Download the above files and execute the following commands
\begin{lstlisting}
$ gcc 2.4.c
$ ./a.out
\end{lstlisting}
\item Given that 
\begin{align}
p_{X}\brak{x} = \frac{1}{\sqrt{2\pi}}\exp\brak{-\frac{x^2}{2}}, -\infty < x \infty,
\end{align}
repeat the above exercise theoretically.\\
\solution\\
\begin{align}
    E\sbrak{X} &= \int_{-\infty}^{\infty}x p_{X}\brak{x} dx \\
    E\sbrak{X} &= \int_{-\infty}^{\infty}x
    \frac{1}{\sqrt{2\pi}}\exp\brak{-\frac{x^2}{2}} dx    
\end{align}
Taking $\frac{x^2}{2} = t$,
\begin{align}
    E\sbrak{X} &= -\int_{\infty}^{\infty}
    \frac{1}{\sqrt{2\pi}}\exp\brak{-t} dt\\
    E\sbrak{X} &= 0
\end{align}
To calculate variance,
\begin{align}
    var\sbrak{X} &= E\sbrak{(X - E\sbrak{X})^2}\\
    var\sbrak{X} &= E\sbrak{X^2}\\
    var\sbrak{X} &= \int_{-\infty}^{\infty}x^2 p_{X}\brak{x} dx \\
    var\sbrak{X} &= \int_{-\infty}^{\infty}x^2
    \frac{1}{\sqrt{2\pi}}\exp\brak{-\frac{x^2}{2}} dx 
\end{align}
We know that,
\begin{align}
    \int_{-\infty}^{\infty}x^2\exp\brak{-\frac{x^2}{2}} &= \sqrt{2\pi}\\
    var\sbrak{X} &= 1
\end{align}
\begin{figure}[h!]
    \centering
    \includegraphics[width=\columnwidth]{./figs/2.3.png}
    \caption{PDF of $X$}
    \label{fig:my_label}
\end{figure}
\begin{figure}[h!]
    \centering
    \includegraphics[width=\columnwidth]{./figs/2.2.png}
    \caption{CDF of $X$}
    \label{fig:my_label}
\end{figure}
\newpage
The Python plot for the CDF and PDF of $X$ are
\begin{lstlisting}
https://github.com/Killomanny/Assignments/tree/main/Assignmets2.0/codes/2.5_1.py
https://github.com/Killomanny/Assignments/tree/main/Assignmets2.0/codes/2.5_2.py
\end{lstlisting}
\end{enumerate}
\section{From uniform to other}
\begin{enumerate}[label=\thesection.\arabic*
,ref=\thesection.\theenumi]
\item
Generate samples of 
%
\begin{equation}
V = -2\ln\brak{1-U}
\end{equation}
%
and plot its CDF.  \\
\solution Download the following files and execute the  C program.
\begin{lstlisting}
wget https://github.com/Killomanny/Assignments/tree/main/Assignmets2.0/codes/3.1.c
wget https://github.com/Killomanny/Assignments/tree/main/Assignmets2.0/codes/source.h
\end{lstlisting}
Download the above files and execute the following commands
\begin{lstlisting}
$ gcc 3.1.c -lm
$ ./a.out
\end{lstlisting}
The CDF of $V$ is plotted in Fig. \ref{fig:3.1} using the code below
\begin{lstlisting}
wget https://github.com/Killomanny/Assignments/tree/main/Assignmets2.0/codes/3.1pyth.py
\end{lstlisting}
Download the above files and execute the following commands to produce Fig.\ref{fig:3.1}
\begin{lstlisting}
$ python3 3.1pyth.py
\end{lstlisting}
\begin{figure}[!h]
\centering
\includegraphics[width=\columnwidth]{./figs/3.1.png}
\caption{The CDF of $V$}
\label{fig:3.1}
\end{figure}
\item Find a theoretical expression for $F_V(x)$.\\
\solution
If Y = g(X), we know that $F_Y(y) = F_X(g^{-1}(y))$, here 
\begin{align}
V &= -2\ln{(1-U)} \\
1-U &= e^{\frac{-V}{2}}\\
U &= 1 - e^{\frac{-V}{2}} \\ 
F_V(x) &= F_U(1 - e^{\frac{-x}{2}}) 
\end{align}
 \begin{align}
\implies
  F_V(x)=
  \begin{cases}
   0                         & x < 0 \\
	1 - e^{\frac{-x}{2}} & x \geq 0
	\end{cases}
 \end{align}
%\item
%Generate the Rayleigh distribution from Uniform. Verify your result through graphical plots.
\end{enumerate}

\section{Triangular Distribution}
\begin{enumerate}[label=\thesection.\arabic*
,ref=\thesection.\theenumi]
    \item Generate
    \begin{align}
        T = U_1 + U_2
    \end{align}
\solution\\
Download and run the following C code to generate tri.dat file.
\begin{lstlisting}
wget https://github.com/Killomanny/Assignments/blob/main/Assignmets2.0/codes/4.1.c
\end{lstlisting}
\item Find the CDF of $T$.\\
\solution
\begin{figure}[h!]
    \centering
    \includegraphics[width=\columnwidth]{figs/4.2.png}
    \caption{CDF of $T$}
    \label{fig:my_label}
\end{figure}
\clearpage
The following code plots the CDF of $T$
\begin{lstlisting}
wget https://github.com/Killomanny/Assignments/blob/main/Assignmets2.0/codes/4.2.py
\end{lstlisting}
\item Find the PDF of $T$.
\solution
\begin{figure}[h!]
    \centering
    \includegraphics[width=\columnwidth]{figs/4.3.png}
    \caption{PDF of $T$}
    \label{fig:my_label}
\end{figure}
The following code plots the PDF of $T$
\begin{lstlisting}
wget https://github.com/Killomanny/Assignments/blob/main/Assignmets2.0/codes/4.3.py
\end{lstlisting}
\item Find the theoritical expressions for PDF and CDF of $T$.\\
\solution\\
\begin{align}
    p_T(x) &= p_{U_1 + U_2}(x) = p_{U_1}(x) * p_{U_2}(x)\\
    p_T(x) &= \int_{-\infty}^{\infty}p_{U_1}(\tau)p_{U_2}(x - \tau)\\
    p_T(x) &= \int_0^1p_{U_2}(x - \tau)
\end{align}
\begin{align}
    \displaystyle p_T(x) = \begin{cases} 
    0 & \text{$x \leq 0$} \\  
    \int_0^x 1d\tau & \text{$0 < x < 1$} \\  
    \int_{x - 1}^1 1d\tau & \text{$1 \leq x < 2$} \\
    0 & \text{$x > 2$}
    \end{cases}
\end{align}
\begin{align}
    \displaystyle p_T(x) = \begin{cases} 
    0 & \text{$x \leq 0$} \\  
    x & \text{$0 < x < 1$} \\  
    2 - x & \text{$1 \leq x < 2$} \\
    0 & \text{$x > 2$}
    \end{cases}
\end{align}
Expression for CDF can be obtained by integrating $p_T(x)$ w.r.t. $X$
\begin{align}
    \displaystyle F_T(x) = \begin{cases} 
    0 & \text{$x \leq 0$} \\  
    \frac{x^2}{2} & \text{$0 < x < 1$} \\  
    -\frac{x^2}{2} + 2x - 1 & \text{$1 \leq x < 2$} \\
    1 & \text{$x > 2$}
    \end{cases}
\end{align}
\item Verify the results through a plot.\\
\solution\\
\begin{figure}[h!]
    \centering
    \includegraphics[width=\columnwidth]{figs/4.5pdf.png}
    \caption{Theoretical PDF of $T$}
    \label{fig:my_label}
\end{figure}
\begin{figure}[h!]
    \centering
    \includegraphics[width=\columnwidth]{figs/4.5cdf.png}
    \caption{The CDF of $T$}
    \label{fig:my_label}
\end{figure}
PDF and CDF plotted by the Python codes
\begin{lstlisting}
wget https://github.com/Killomanny/Assignments/blob/main/Assignmets2.0/codes/4.5pdf.py
wget https://github.com/Killomanny/Assignments/blob/main/Assignmets2.0/codes/4.5cdf.py
\end{lstlisting}
\end{enumerate}

\end{document}
