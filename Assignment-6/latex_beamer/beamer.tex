\let\negmedspace\undefined
\let\negthickspace\undefined
\documentclass[journal,12pt,twocolumn]{IEEEtran}

\usepackage{csquotes}
\usepackage{comment}
\usepackage{enumerate}
\usepackage{amsmath,amssymb,amsthm}
\usepackage{graphicx}
\let\vec\mathbf
\newcommand{\myvec}[1]{\ensuremath{\begin{pmatrix}#1\end{pmatrix}}}
\providecommand{\brak}[1]{\ensuremath{\left(#1\right)}}
\providecommand{\pr}[1]{\ensuremath{\Pr\left(#1\right)}}



\title{Assignment 6 ( Cbse 12 ex 13.2 12 )}
\author{Busireddy Asli Nitej Reddy (CS21BTECH11011)}
\date{}



\begin{document}



\maketitle



\begin{abstract}
This document contains the solution to Cbse 12 ex 13.2 12
\end{abstract}

\section*{\textbf{Problem}}
 Assume that each born child is equally likely to be a boy or a girl. If a family has two children, what is the conditional probability that both are girls given that

     \begin{enumerate}[i)]
         \item  the youngest is a girl
         \item  at least one is a girl
     \end{enumerate}
     

\section*{\textbf{Solution}}
    
    
    Let the random variables $X_{i}$ map to the set $\{ 0,1 \}$ as described in Table
    where Event A denotes first child or older child is girl and Event B denotes second child or youngest is girl and also Event C where at least one girl is born 


    \begin{table}[!htb]
        \centering
         {
            \begin{tabular}{|c|c|}
              \hline
               variable & event \\ \hline
               $X_{1}$ = 1  & A  \\ \hline
               $X_{2}$ = 1  & B  \\ \hline
               $X_{3}$ = 1  & C  \\ \hline
            \end{tabular}
            }
        \caption{ Random variables }
        \label{tab:table}
    \end{table}
    
    as given girl and boy are likely to born so the below probability can be written 
    
    \begin{align}
        \pr{ X_{1} = 1 } &= \frac{1}{2} \\
        \pr{ X_{2} = 1 } &= \frac{1}{2} \\
        \pr{ X_{1} = 1 , X_{2} = 1 } &= \frac{1}{4}
    \end{align}
    
    as both events are independent because one doesn't depend on other so
   
   

    \begin{enumerate}
    
    
    
        \item[(i)]
        
         here the youngest is a girl and let us denote the event as C where conditional probability that both are girls given that youngest is a girl
         
         \begin{align}
             \pr{C} &= \pr{ (X_{1} = 1,X_{2} = 1) | X_{2} = 1 } \\
                    &= \frac{\pr{ {X_{1} = 1,X_{2} = 1}}}{\pr{X_{2} = 1}} \\
                    &= \pr{X_{1} = 1} \\
                    &= \frac{1}{2}
         \end{align}
         
         $\therefore$ the conditional probability that both are girls given that the youngest is a girl is $0.5$
         
         
         
        \item[(ii)]
        
        here given that at least one girl is born find conditional probability that both are girls lets us denote that event by D as is given by
        
        \begin{align}
            \pr{D} &= \pr{ (X_{1} = 1,X_{2} = 1) | X_{3} = 1 } \\
                   &= \frac{\pr{ X_{1} = 1,X_{2} = 1,X_{3} = 1 }}{\pr{X_{3} = 1}} 
                   \label{eq:eq1}
                   \\
                   &= \frac{\pr{ X_{1} = 1,X_{2} = 1}}{\pr{X_{3} = 1}}
                   \label{eq:eq2}
        \end{align}
        
        Eq\eqref{eq:eq2} numerator is same as Eq\eqref{eq:eq1} numerator because the event where both are gorls is same as both are girls and alleast one girl
        
        \begin{align}
           \pr{X_{3} = 1 } &= 1 - \pr{ X_{1}=0 , X_{2} = 0 } \\
                           &= 1 - \frac{1}{4} \\
                           &= \frac{3}{4}
        \end{align}
        
        on substituting values in Eq\eqref{eq:eq2} we will get 
        
        \begin{align}
            \pr{D} &= \frac{ \frac{1}{4} }{ \frac{3}{4} } \\
                   &= \frac{1}{3}
        \end{align}
        
        $\therefore$ the conditional probability that both are girls given that the there is atleast one girl is born is $\frac{1}{3}$
        
        
        
    \end{enumerate}



\end{document}
