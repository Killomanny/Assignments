\documentclass{beamer}
\usepackage{listings}


\lstset{
%language=C,
frame=single, 
breaklines=true,
columns=fullflexible
}


\usepackage{subcaption}
\usepackage{url}
\usepackage{tikz}
\usepackage{tkz-euclide} 
\usetikzlibrary{calc,math}
\usepackage{float}
\usepackage[export]{adjustbox}
\usepackage[utf8]{inputenc}
\usepackage{amsmath}
\usepackage[version=4]{mhchem}


\newcounter{saveenumi}


\newcommand\norm[1]{\left\lVert#1\right\rVert}
\newcommand{\myvec}[1]{\ensuremath{\begin{pmatrix}#1\end{pmatrix}}}
\newcommand{\seti}{\setcounter{saveenumi}{\value{enumi}}}
\newcommand{\conti}{\setcounter{enumi}{\value{saveenumi}}}


\providecommand{\brak}[1]{\ensuremath{\left(#1\right)}}
\providecommand{\cbrak}[1]{\ensuremath{\left\{#1\right\}}}
\providecommand{\lcbrak}[1]{\ensuremath{\left\{#1\right.}}
\providecommand{\rcbrak}[1]{\ensuremath{\left.#1\right\}}}
\providecommand{\pr}[1]{\ensuremath{\Pr\left(#1\right)}}
\providecommand{\sbrak}[1]{\ensuremath{{}\left[#1\right]}}
\providecommand{\lsbrak}[1]{\ensuremath{{}\left[#1\right.}}
\providecommand{\rsbrak}[1]{\ensuremath{{}\left.#1\right]}}


\renewcommand{\vec}[1]{\mathbf{#1}}


\usetheme{Boadilla}


\title{ Assignment-11}
\author{Asli}
\institute{IIT Hyderabad}
\date{\today }



\begin{document}



\begin{frame}
      \titlepage
\end{frame}



\begin{frame}{Abstract}
     \begin{block}{Abstract}
          This document contains the solution to Chapter 8 Exercise problem 8.19 from Papoulis Book
      \end{block}
\end{frame}



\begin{frame}{Question}
      \begin{block}{Question}
           The readings of a voltmeter introduces an error $nu$ with mean $0$. We wish to estimate its standard deviation $\sigma$. We measure a calibrated source V = 3 V four times and obtain the values $ 2.90, 3.15 , 3.05, 2.96 $ Assuming that $\nu$ is normal, find the 0.95 confidence interval of $\sigma$
      \end{block}
\end{frame}



\begin{frame}{Solution}
       \begin{block}{Solution}
        so here there are 4 observations like  $ 2.90, 3.15 , 3.05, 2.96 $ where expected values for each are $ 3.00 $ \\
        Also $ 0.95 $ level of confidence for $\sigma$ is nothing but an interval between $ 0.025 , 0.975 $ 
       \end{block}
\end{frame}



\begin{frame}{Formulae}
    \begin{block}{Formulae1}
            The Confidence interval for the variance is given by:
            
               \begin{align}
                   \frac{k}{\chi^{2}_{0.025}} > \sigma^{2} > \frac{k}{\chi^{2}_{0.975}}
                   \label{form_1}
               \end{align}
               
    \end{block}
\end{frame}



\begin{frame}{Formulae}
    \begin{block}{Formulae2}
         $\chi^{2}_{0.025}$ and $\chi^{2}_{0.975}$ can be calculated respectively from Fig\ref{fig:my_label1} and Fig\ref{fig:my_label2} for values of v = $4$ and the critical probability from above 
         
         \begin{align}
             \chi^{2}_{0.025} &= 0.484  \\
             \chi^{2}_{0.975} &= 11.143
         \end{align}
         
    \end{block}
\end{frame}



\begin{frame}{Formulae}
    \begin{block}{Formulae3}
       the value of k is given by n$\times$v \\
       where v is the variance of the observations and n is the no of observations
       
       \begin{align*}
           k= 4\brak{ (2.90 - 3.00)^{2} + (3.15 - 3.00)^{2} + (3.05 - 3.00)^{2} + (2.96 - 3.00)^{2} }
       \end{align*}
       
       which on calculating we will get 
       
       \begin{align}
           k = 0.0366
       \end{align}
       
    \end{block}
\end{frame}



\begin{frame}

     \begin{figure}
           \centering
           \includegraphics[width= 10cm, height=7cm]{fig1.png}
           \caption{ lower tail critical values of ${\chi}^{2}$ with v degrees of freedom }
           \label{fig:my_label1}
      \end{figure}
      
\end{frame}



\begin{frame}

     \begin{figure}
           \centering
           \includegraphics[width= 10cm, height=7cm]{fig2.png}
           \caption{{ lower tail critical values of ${\chi}^{2}$ with v degrees of freedom }}
           \label{fig:my_label2}
      \end{figure}
      
\end{frame}



\begin{frame}{Substituting and solving}
     \begin{block}{Solving}
        On substituting all values in Eq\eqref{form_1} we will get
        
        \begin{align}
            \frac{0.0366}{0.484} > \sigma^{2} > \frac{0.0366}{11.143}
            \label{form_2}
        \end{align}
        
        on simplyfying Eq\eqref{form_2} we will get
        
        \begin{align}
           0.275  > \sigma > 0.057 
        \end{align}
        
        or simply
        
        \begin{align}
            0.057 < \sigma < 0.275
        \end{align}
        
     \end{block}
\end{frame}



\end{document}
